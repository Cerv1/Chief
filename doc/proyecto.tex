\documentclass[tikz,a4paper,11pt]{book}
%\documentclass[a4paper,twoside,11pt,titlepage]{book}
\usepackage{listings}
\usepackage[utf8]{inputenc}
\usepackage{csquotes}
\usepackage[spanish,es-noquoting]{babel}
\usepackage{forest}
\usepackage{xcolor}
\usepackage[export]{adjustbox}
\usepackage{caption}
\usepackage{float}
\usetikzlibrary{arrows.meta,shapes,positioning,trees}

% \usepackage[style=list, number=none]{glossary} %
%\usepackage{titlesec}
%\usepackage{pailatino}

\decimalpoint
\usepackage{dcolumn}
\newcolumntype{.}{D{.}{\esperiod}{-1}}
\makeatletter
\addto\shorthandsspanish{\let\esperiod\es@period@code}
\makeatother


%\usepackage[chapter]{algorithm}
\RequirePackage{verbatim}
%\RequirePackage[Glenn]{fncychap}
\usepackage{fancyhdr}
\usepackage{graphicx}
\usepackage{afterpage}
\usepackage{longtable}
\usepackage[pdfborder={000}]{hyperref} %referencia


% ********************************************************************
% Re-usable information
% ********************************************************************
\newcommand{\myTitle}{Chief\xspace}
\newcommand{\myDegree}{Grado en Ingeniería Informática\xspace}
\newcommand{\myName}{Sergio Cervilla Ortega\xspace}
\newcommand{\myProf}{Juan Julián Merelo Guervós\xspace}
\newcommand{\myFaculty}{Escuela Técnica Superior de Ingenierías Informática y de Telecomunicación\xspace}
\newcommand{\myFacultyShort}{E.T.S. de Ingenierías Informática y de Telecomunicación\xspace}
% \newcommand{\myDepartment}{Departamento de Arquitectura y Tecnología de Computadores\xspace}
\newcommand{\myUni}{\protect{Universidad de Granada}\xspace}
\newcommand{\myLocation}{Granada\xspace}
\newcommand{\myTime}{\today\xspace}
\newcommand{\myVersion}{Version 1.0\xspace}
\newcommand{\innerxsep}{5mm}

\hypersetup{
pdfauthor = {\myName (cervick13@gmail.com)},
pdftitle = {\myTitle},
pdfsubject = {},
pdfcreator = {LaTeX con el paquete ....},
pdfproducer = {pdflatex}
}

%\hyphenation{}


%\usepackage{doxygen/doxygen}
%\usepackage{pdfpages}
\usepackage{url}
\usepackage{hyperref}
\usepackage{colortbl,longtable}
\usepackage[stable]{footmisc}
%\usepackage{index}

%\makeindex
%\usepackage[style=long, cols=2,border=plain,toc=true,number=none]{glossary}
% \makeglossary

% Definición de comandos que me son tiles:
%\renewcommand{\indexname}{Índice alfabético}
%\renewcommand{\glossaryname}{Glosario}

\pagestyle{fancy}
\fancyhf{}
\fancyhead[LO]{\leftmark}
\fancyhead[RE]{\rightmark}
\fancyhead[RO,LE]{\textbf{\thepage}}


%Para conseguir que en las páginas en blanco no ponga cabecerass
\makeatletter
\def\clearpage{%
  \ifvmode
    \ifnum \@dbltopnum =\m@ne
      \ifdim \pagetotal <\topskip
        \hbox{}
      \fi
    \fi
  \fi
  \newpage
  \thispagestyle{empty}
  \write\m@ne{}
  \vbox{}
  \penalty -\@Mi
}
\makeatother


\usepackage{pdfpages}
\usepackage{fourier}
\begin{document}
\input{portada/portada}
\chapter*{}
\thispagestyle{empty}

\begin{center}
{\large\bfseries Chief \\ Apoyo a la labor de la Policía Local}\\
\end{center}
\begin{center}
Sergio Cervilla Ortega\\
\end{center}

%\vspace{0.7cm}

\noindent{\textbf{Resumen}}\\

Lorem ipsum dolor sit amet, consectetur adipisicing elit, sed do eiusmod tempor incididunt ut labore et dolore magna aliqua. Ut enim ad minim veniam, quis nostrud exercitation ullamco laboris nisi ut aliquip ex ea commodo consequat. Duis aute irure dolor in reprehenderit in voluptate velit esse cillum dolore eu fugiat nulla pariatur. Excepteur sint occaecat cupidatat non proident, sunt in culpa qui officia deserunt mollit anim id est laborum.

\cleardoublepage

\thispagestyle{empty}

\noindent\rule[-1ex]{\textwidth}{2pt}\\[4.5ex]

D. \textbf{Juan Julián Merelo Guervós}, Profesor del Área de XXXX del Departamento de Arquitectura y Tecnología de Computadores de la Universidad de Granada.

\vspace{0.5cm}

\textbf{Informo:}

\vspace{0.5cm}

Que el presente trabajo, titulado \textit{\textbf{Chief, blah blah}},
ha sido realizado bajo mi supervisión por \textbf{Sergio Cervilla Ortega}, y autorizo la defensa de dicho trabajo ante el tribunal
que corresponda.

\vspace{0.5cm}

Y para que conste, expiden y firman el presente informe en Granada a X de mes de 2018.

\vspace{1cm}

\textbf{El director: }

\vspace{5cm}

\noindent \textbf{Juan Julián Merelo Guervós}

\frontmatter
\tableofcontents
\listoffigures
%\listoftables
%\mainmatter
%\setlength{\parskip}{5pt}

\chapter{Introducción}

Vivimos cada día en un mundo más conectado a Internet, es una época en la que se está dejando de 
lado el lápiz y el papel. Donde se está abriendo paso la tecnología. Estos nuevos medios, bien
usados, son capaces de facilitarnos en numerosas ocasiones las tareas del día a día. Convierte acciones
tediosas en simples gestos que, con el tiempo, acabamos automatizando y realizamos sin esfuerzo. 
Bajo dicha motivación, decidí crear un sistema informatizado para ayudar en las labores del cuerpo policial de Maracena.\\

Previamente, el cuerpo policial utilizaba un esquema bastante anticuado para desemplear sus funciones.
Por ejemplo, los registros de incidencias se iban apuntando manualmente y al final del turno
se reunían para poner en común todos los datos en un documento que posteriormente se guardaba.
Y al igual que en el ejemplo anterior, debían esperar a llegar a las dependencias para poder 
rellenar cualquier tipo de denuncia administrativa con los datos que habían apuntado en 
un papel para no olvidarlos.\\

Para ello, he diseñado una aplicación web que se encarga de gestionar una gran cantidad de 
modelos de denuncia y que permite que se rellenen de forma segura, a prueba de 
fallos y de una manera muy intuitiva para el usuario. Además de poder gestionar dichos documentos,
se ha creado un sistema de registro de incidencias completamente automatizado para que en 
ningún momento se pierda ningún dato y un gran número de funcionalidades más que se explicarán en 
posteriores puntos. El Ayuntamiento de Maracena será pionero en este ámbito, ya que han sido los 
primeros que han apostado por las nuevas tecnologías. Una gran apuesta, pero que traerá grandes
beneficios a la calidad del trabajo del cuerpo policial y a la seguridad de los ciudadanos. \\

Este Trabajo de Fin de Grado ha sido realizado completamente por mi, Sergio 
Cervilla Ortega durante el año 2018 y es libre bajo la licencia GNU General Public License v3.0 \cite{gplv3}\\ 

Ha sido desarrollado bajo el permiso, aceptación, revisión y ayuda del Ayuntamiento de Maracena, localidad en
la que resido a fecha de hoy.\\


\chapter{Tecnologías}
\section{Nodejs}


\input{secciones/eleccion_dms}
% \input{secciones/servicio_mensajeria}
% \input{secciones/planificacion}
% \input{secciones/implementacion}
% \input{secciones/conclusiones}

%
\nocite{*}
\bibliography{bibliografia}\addcontentsline{toc}{chapter}{Bibliografía}
\bibliographystyle{unsrt}


\end{document}
