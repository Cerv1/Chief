\chapter{Introducción}

Vivimos cada día en un mundo más conectado a Internet. Es una época en la que se está dejando de 
lado el lápiz y el papel, dejando las puertas abiertas a la tecnología. Estos nuevos medios, bien
usados, son capaces de facilitarnos en numerosas ocasiones las tareas del día a día. Convierte acciones
tediosas en simples gestos que, con el tiempo, acabamos automatizando y realizamos sin esfuerzo.\\ 

De esta manera conseguiría mejorar la calidad del servicio de los agentes, ya que de por sí es un 
trabajo duro. Repleto de emociones y actividad física. Bajo este contexto, lo último que desea un
agente a la hora de finalizar su turno es reescribir todo el trabajo realizado a lo largo de un 
turno para pasarlo de una hoja de papel a otra, pero esta última, digital.\\ 

Bajo dicha motivación, decidí crear un sistema informatizado para ayudar en las labores del cuerpo 
policial de Maracena. Una vez decidido el proyecto, procedí a reunirme con el Ayuntamiento de la 
localidad para escuchar las sugerencias de los  concejales y los policías. Así, conseguiría conocer de 
primera mano  sus inquietudes y qué les gustaría ver en el software a desarrollar.\\

La filosofía seguida en el desarrollo de dicho proyecto es una filosofía completamente \textit{open source}.
Por tanto, todo el software utilizado, así como las herramientas empleadas en su desarrollo son completamente
gratuitas y de código libre.\\ 

% TO DO: ¿Incluir?
%Bajo mi punto de vista se debería de apostar más por este tipo de desarrollos
%ya que el software libre es bueno para la comunidad de desarrolladores y al ser transparente permite que 
%cualquiera pueda mirar el código para mejorarlo o para asegurarse de que hacen lo que dicen.\\

En este documento se presenta de manera ordenada, clara y concisa los pasos que se han seguido para
la elaboración de dicho proyecto. Ha sido realizado completamente por mi, Sergio Cervilla Ortega durante 
el año 2018 y es libre bajo la licencia GNU General Public License v3.0 \cite{gplv3}\\ 


