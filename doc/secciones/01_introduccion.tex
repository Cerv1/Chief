\chapter{Introducción}

Vivimos cada día en un mundo más conectado a Internet. Es una época en la que se está dejando de 
lado el lápiz y el papel, abriendo las puertas de par en par a la tecnología. Estos nuevos medios, bien
usados, son capaces de facilitarnos en numerosas ocasiones las tareas del día a día. Convierte acciones
tediosas en simples gestos que, con el tiempo, acabamos automatizando y realizamos sin esfuerzo.\\ 

\textbf{Chief} busca facilitar la labor a los agentes de policía a través del uso de la informática y las nuevas tecnologías. De esta manera, se conseguirá una mejora en la calidad del servicio de los agentes, ya que de por sí es un 
trabajo duro. Repleto de emociones y actividad física. Bajo este contexto, lo último que desea un
agente a la hora de finalizar su turno es reescribir todo el trabajo realizado a lo largo de un 
turno para pasarlo de una hoja de papel a otra, pero esta última, digital.\\ 

Bajo dicha motivación, decidí crear un sistema informatizado para facilitar las labores del cuerpo 
policial de Maracena. Una vez decidido el proyecto, procedí a reunirme con el Ayuntamiento de la 
localidad para escuchar las sugerencias de los  concejales y los policías. Así, conseguiría conocer de 
primera mano  sus inquietudes y qué les gustaría ver en el software a desarrollar.\\

La filosofía seguida en el desarrollo de dicho proyecto es una filosofía completamente \textit{open source}.
Por tanto, todo el software utilizado, así como las herramientas empleadas en su desarrollo son completamente
gratuitas y de código libre.\\ 

En este documento se presenta de manera ordenada, clara y concisa los pasos que se han seguido para
la elaboración de dicho proyecto, desde su concepción hasta la implementación. Ha sido realizado completamente por mi, Sergio Cervilla Ortega durante 
el año 2018 y es libre bajo la licencia GNU General Public License v3.0 \cite{gplv3}\\ 


