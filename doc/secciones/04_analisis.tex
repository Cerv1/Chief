\chapter{Análisis del problema}
 
 Como todo software, esta aplicación se ha diseñado y desarrollado para cubrir una
 necesidad. En base a tal necesidad, podemos crear una descripción completa de los 
 actores así como una lista de requisitos completa con la que se cubran los objetivos 
 propuestos.

\section{Descripción de los actores}
Vamos a disponer de dos actores: el \textbf{usuario} y el \textbf{administrador}. 

El \textbf{usuario} será el agente de policía que desée realizar cualquiera de las acciones
disponibles en la aplicación. Este actor no tiene por qué tener ninguna experiencia previa
con aplicaciones web pero en este escenario se les ha formado con unas nociones básicas 
a modo de tutorial de como realizar todas las acciones posibles en la aplicación y las consecuencias
que tiene en el servidor.\\

El \textbf{administrador} será la persona encargada de asegurar el correcto funcionamiento 
del software así como el encargado de la gestión de los datos de usuario y de la aplicación. Este
actor, por tanto, debe tener un alto conocimiento de las tecnologías con las que se ha construido
\textbf{Chief} para poder dar una rápida respuesta a los posibles problemas del usuario.

\section{Análisis de requisitos}

Los requisitos serán divididos en 3 tipos:

\begin{enumerate}
   \item \textbf{}
\end{enumerate}


\subsection{Requisitos funcionales}

asdadasdasdasdasds
\subsection{Requisitos no funcionales}
\subsection{Requisitos de información}


\section{Análisis de las soluciones}

\section{Solucion propuesta}

\section{Análisis de seguridad}
