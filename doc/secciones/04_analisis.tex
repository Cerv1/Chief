\chapter{Análisis del problema}
 
 Como todo software, esta aplicación se ha diseñado y desarrollado para cubrir una
 necesidad. En base a tal necesidad, podemos crear una descripción completa de los 
 actores así como una lista de requisitos completa con la que se cubran los objetivos 
 propuestos.

\section{Descripción de los actores}
Vamos a disponer de dos actores: el \textbf{usuario} y el \textbf{administrador}.\\

El \textbf{usuario} será el agente de policía que desée realizar cualquiera de las acciones
disponibles en la aplicación. Este actor no tiene por qué tener ninguna experiencia previa
con aplicaciones web pero en este escenario se les ha formado con unas nociones básicas 
a modo de tutorial de como realizar todas las acciones posibles en la aplicación y las consecuencias
que tiene en el servidor.\\

El \textbf{administrador} será la persona encargada de asegurar el correcto funcionamiento 
del software así como el encargado de la gestión de los datos de usuario y de la aplicación. Este
actor, por tanto, debe tener un alto conocimiento de las tecnologías con las que se ha construido
\textbf{Chief} para poder dar una rápida respuesta a los posibles problemas del usuario.

\section{Análisis de requisitos}

Los requisitos serán divididos en 3 tipos:

\begin{itemize}
   \item \textbf{Requisitos funcionales:} Son servicios que el sistema debe proporcionar, cómo
   debería responder a entradas concretas y como debe reaccionar el sistema en situaciones 
   particulares. En algunos casos, se puede especificar explícitamente qué no debe hacer el sistema.

   \item \textbf{Requisitos no funcionales:} Son requisitos que no tienen que estar directamente relacionado
   con el funcionamiento de la aplicación sino más bien, con el proceso del desarrollo.
   
   \item \textbf{Requisitos de información:} Estos requisitos están relacionados con la información 
   que se va a guardar en el sistema.

   % Definiciones por: 
   % Chapter 6. Ian Sommerville (2006). Software Engineering, 8th ed.. ISBN 0-321-31379-8. 
\end{itemize}

Para el siguiente punto se utilizarán las siguientes abreviaturas:\\  
\begin{table}[H]
   \centering
   \label{tabla-abreviaturas}
   \begin{tabular}{|c|l|}
   \hline
   \textbf{Abreviatura} & \textbf{Significado} \\ \hline
   R.F X       & Para denotar el requisito funcional número \textit{X}.\\ \hline
   R.F X-Y     & Para denotar el subrequisito funcional número \textit{Y} de \textit{X}\\ \hline
   R.N.F X     & Para denotar el requisito no funcional número \textit{X}\\ \hline
   R.N.F X-Y   & Para denotar el subrequisito no funcional número \textit{Y} de \textit{X}\\ \hline
   R.I X       & Para denotar el requisito de información número \textit{X}\\ \hline
   R.I X-Y     & Para denotar el subrequisito de información número \textit{Y} de \textit{X}\\ \hline
   \end{tabular}
   \caption{Tabla de abreviaturas para los tipos de requisitos.}
\end{table}


\subsection{Requisitos funcionales}
Los requisitos funcionales de la aplicación son los siguientes: 

\begin{itemize}
	\item \textbf{R.F 1}. Administración de los agentes.
	\begin{itemize}
		\item \textbf{R.F 1.1}. Registro de los agentes.
		\item \textbf{R.F 1.2}. Acceso de los agentes.
		\item \textbf{R.F 1.2}. Cerrar sesión.
	\end{itemize}

	\item \textbf{R.F 2}. Administración de turnos.
	\begin{itemize}
		\item \textbf{R.F 2.1}. Crear un documento de turno.
		\item \textbf{R.F 2.2}. Crear una nueva incidencia.
		\item \textbf{R.F 2.3}. Finalizar un parte de servicio.
	\end{itemize}

	\item \textbf{R.F 3}. Creación de ordenanzas.
	\begin{itemize}
		\item \textbf{R.F 3.1}. Crear una ordenanza de limpieza.
		\item \textbf{R.F 3.2}. Crear una ordenanza de ruidos.
		\begin{itemize}
			\item \textbf{R.F 3.2.1}. Crear un acta de molestias de ruidos en la vía pública.
			\item \textbf{R.F 3.2.2}. Crear un acta de molestias de ruidos en la vía domicilio.
			\item \textbf{R.F 3.2.3}. Crear un acta de molestias de ruidos en la vía local.
			\item \textbf{R.F 3.2.4}. Crear un acta de medición de ruidos.
		\end{itemize}
		\item \textbf{R.F 3.3}. Crear una ordenanza de obras.
		\begin{itemize}
			\item \textbf{R.F 3.3.1}. Crear un acta inspección de obras.
		\end{itemize}
		\item \textbf{R.F 3.4}. Crear una ordenanza de residuos.
		\begin{itemize}
			\item \textbf{R.F 3.3.1}. Crear un acta hallazgo de residuos.
		\end{itemize}
	\end{itemize}

	\item \textbf{R.F 4}. Creación de partes de accidentes.
	\begin{itemize}
		\item \textbf{R.F 4.1}. Crear parte de accidente de 2 vehículos.
		\item \textbf{R.F 4.2}. Crear parte de accidente de 3 vehículos.
	\end{itemize}

	\item \textbf{R.F 5}. Administrar croquis.
	\begin{itemize}
		\item \textbf{R.F 5.1}. Crear un croquis.
		\item \textbf{R.F 5.2}. Enviar un croquis.
	\end{itemize}


\end{itemize}

\subsection{Requisitos no funcionales}
\subsection{Requisitos de información}


\section{Análisis de las soluciones}

\section{Solucion propuesta}

\section{Análisis de seguridad}
