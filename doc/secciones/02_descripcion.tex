\chapter{Descripción del problema}

En la actualidad, el cuerpo policial utiliza un esquema bastante anticuado para ejercer sus funciones y las convertía en tareas más complicadas. Debido a que tienen 
que manejar de manera precisa una gran cantidad de documentos con esta metodología, se consigue una peor gestión del tiempo por parte del agente.\\

Por ejemplo, los registros de incidencias se apuntaban manualmente y al final del turno, todos los agentes se reunían para poner 
en común los datos recopilados en un documento, que posteriormente se guardaba en los ordenadores de la jefatura. Del mismo modo que en el ejemplo anterior, debían 
esperar a llegar a las dependencias para poder rellenar cualquier tipo de denuncia administrativa con los datos que habían
apuntado en un papel para que no se olvidaran.\\

Para ello, he diseñado una \textbf{aplicación web} que se encarga de gestionar una gran cantidad de 
modelos de denuncia, permitiendo que se rellenen de forma segura, a prueba de 
fallos y de una manera muy intuitiva para el usuario. Además de poder gestionar dichos documentos,
se ha creado un sistema de registro de incidencias completamente automatizado para que no se pierdan datos en ningún momento. Además de dicho sistema, se han creado un gran número de funcionalidades para ayudar a la labor de los agentes, que se explicarán en 
posteriores puntos. El Ayuntamiento de Maracena será pionero en este ámbito, ya que han sido los 
primeros que han elegido actualizarse mediante el uso de las nuevas tecnologías y una aplicación completamente libre. Una gran apuesta, pero que traerá grandes
beneficios a la calidad del trabajo del cuerpo policial y a la seguridad de los ciudadanos en su día a día. \\

A continuación se enumeran, de una manera general, los apartados que se persiguen a través de este trabajo:

\begin{enumerate}

    \item \textbf{Facilitar el trabajo a los policías locales de Maracena.}\\
    Debido a la inclusión de un sistema completamente informatizado para la elaboración de 
    denuncias administrativas, gestión de incidentes, partes de accidentes y un sistema de croquis. Consiguiendo,
    por tanto, una mejora en la productividad de los agentes y un aumento de la seguridad
    global de los datos almacenados.
    
    \item \textbf{Inclusión del software libre en organismos del estado.}\\
    Dejando a un lado herramientas privativas sobre las que no tenemos un pleno control. Con este tipo de aplicaciones, no tenemos una visión global
    de los datos que están recopilando o analizando. Además, de este modo se consigue fomentar el desarrollo libre. Porque 
    de esta manera cualquier persona puede sumarse al desarrollo y mejora de \textbf{Chief}.

    \item \textbf{Elaboración de un entorno de pruebas real.}\\
    Se persigue la creación de un entorno virtual lo más parecido posible a la 
    realidad para que el uso de la aplicación pueda ser probado antes del despliegue 
    final. Consiguiendo directamente una versión del código más robusta ante la aparición de posibles errores que se solucionarán en dicho contexto.
   	
    \item \textbf{Probar los conocimientos adquiridos a lo largo del grado.}\\
    Creando un sistema completo en el que tendremos que tener en cuenta todos los 
    aspectos técnicos adquiridos en el transcurso de la carrera. Dada la embergadura del
    proyecto, es necesario disponer de una base muy consolidada de los conocimientos adquiridos
    previamente.

    \item \textbf{Aprendizaje de tecnologías punteras en el sector.}\\
    Durante el desarollo del proyecto se ha buscado aprender y utilizar tecnologías en 
    auge que tienen un alto potencial. Esta decisión se toma en base  a la cantidad de gente
    y empresas que las utilizan, la comunidad tan amplia que tienen así como las contínuas
    mejoras que están recibiendo.  

\end{enumerate}

\section{Alcance del proyecto}

El alcance de este proyecto está centrado en llegar a una versión completamente funcional,
pero sin alcanzar la fase de producción debido a las siguientes cuestiones:

\begin{enumerate}
    \item \textbf{Cursos de formación}. Para que la aplicación sea realmente útil será necesario
    que se impartan cursos de formación tanto para los usuarios que utilizarán el sistema
    como para los administradores o gestores de la plataforma y el servidor.

    \item \textbf{Recursos materiales}. En este apartado podemos encontrar problemas como pueden
    ser la compra y configuración de servidores web, mantenimiento de los equipos informáticos, 
    compra de los dispositivos móviles en los que se ejecutará el cliente.
    
    \item \textbf{Recursos humanos}. Como puede ser la contratación del personal encargado
    de la administración de los servidores, administración y gestión de los datos de usuario , un delegado de protección de datos debido a la nueva ley \textbf{GDPR} o equipo de soporte.

\end{enumerate}
