\chapter{Estado del arte}

Actualmente el estado del mercado para este tipo de desarrollos está muy limitado dado que 
la mayoría de los datos que manejan dichos sistemas son de carácter sensible. Por tanto, se 
debe tener un especial cuidado con el tratamiento de los mismos tanto a la hora de transmitirlos
como en su almacenamiento.\\

Esta premisa ha conseguido que la única aplicación actual del mercado, y aún en fase de desarrollo,
utilizada por la Guardia Civil de Tráfico sea de código privativo siendo un modelo de una caja negra
para el usuario. De hecho, está en una fase tan temprana de su desarrollo que aún carece de nombre y 
y las funcionalidades disponibles están muy limitadas.\\ 


\section{Crítica al estado del arte}

Después de un análisis exhaustivo del mercado se puede concluir que este desarrollo va a crear
una aplicación puntera en el sector. Esto es debido a que no existe una aplicación que sea
realmente capaz de facilitar las labores de los agentes del estado en su día a dia. Como se 
ha comentado en el punto anterior, la única aplicación que puede entrar en este mismo sector
además de estar en desarrollo, no es apta para el cuerpo policial sobre el que actúa \textbf{Chief}.\\ 

Por tanto, al tener en cuenta un mayor sector de posibles usuarios, \textbf{Chief} puede conseguir 
una mayor implantación en el mercado. Esto puede provocar que la comunidad de desarrolladores vea
una buena oportunidad y se involucre activamente en el proyecto, consiguiendo mejoras atentiendo a
las necesidades de la comunidad.

\section{Propuesta}
% [hablar de fuera de españa]

% [hablar del open source]