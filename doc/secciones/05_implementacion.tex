\chapter{Implementación}

La implementación del software se ha dividido en hitos. Estos, han sido definidos en Github
y cada uno de ellos contiene un grupo de \textit{issues} que se corresponden con las distintas
mejoras que ha ido incorporando el software a lo largo de su desarrollo.\\

Los hitos han sido los siguientes en orden cronológico:

\begin{enumerate}
	\item \textbf{Permitir registros y acceso de usuarios.}
	\item \textbf{Desarrollo del sistema de incidencias.}
	\item \textbf{Desarrollo del sistema de croquis.}
	\item \textbf{Desarrollo del sistema de tests unitarios.}
	\item \textbf{Integración continua.}
	\item \textbf{Instalación de la aplicación de manera automática.}
\end{enumerate}

Toda la información relacionada con la implementación y el desarrollo se pueden encontrar en:

\begin{itemize}
	\item \href{https://github.com/Cerv1/Chief/milestones}{Lista de hitos.}
	\item \href{https://github.com/Cerv1/Chief/commits}{Lista de commits.}
\end{itemize}

\newpage

\section{Registro y acceso de usuarios.}
El objetivo de este hito es permitir que un usuario agente cree una cuenta en la aplicación 
y posteriormente pueda acceder a la misma si la contraseña y el usuario es válido. \\

Para poder utilizar el servidor de una manera más cómoda y con un mayor número de funcionalidades se ha utilizado
\textbf{Express.js}\cite{express} para crear el servicio web que se ofrece. De esta manera, conseguimos utilizar el 
servidor de una manera más eficiente.\\

Express.js ofrece un software liviano que dota de funcionalidades como negociación de contenido, una mejora en el 
rendimiento del servidor, la posibilidad de seleccionar un motor de plantillas para servir el contenido... En este caso,
se ha configurado para que utilice como motor de plantillas el ya mencionado \textbf{Pug}. De esta manera, cuando el usuario
realice una petición sobre alguna ruta se renderizará dicha plantilla para que el usuario reciba el archivo HTML, el cual
será interpretado por el navegador que esté utilizando.\\ 

Un usuario que quiera acceder a la aplicación o a cualquiera de las funcionalidades que ofrece deberá estar autenticado.
Para administrar de una manera más segura los datos de sesión de un usuario se ha utilizado la librería \textbf{Passport.js}\cite{passport}.
Esta librería se encarga de autenticar las peticiones hacia el servidor a través de una serie de \textit{plugins} conocidos
como \textit{strategies}. Gracias al comportamiento de esta librería podemos gestionar de una manera sencilla el acceso a los usuarios.
En este caso, se ha establecido un inicio de sesión mediante \textbf{usuario} y \textbf{contraseña} definidos por el agente en el 
momento de registrarse.\\

Cuando un agente realiza un registro, sus datos se almacenan \textbf{encriptados} en la base de datos para que de ninguna manera
se puedan alterar o recuperar de forma fraudulenta. El cifrado de dichos datos se realizan con un algoritmo basado en el cifrado
Blowfish\cite{blowfish} al que se le añaden unos bits aleatorios para proteger nuestro cifrado ante un ataque de Tabla Arcoíris\cite{rainbow}.


\section{Desarrollo del sistema de incidencias.}

El objetivo del segundo hito era conseguir un sistema de incidencias completamente funcional. Para considerarse terminado el hito, se debe garantizar que los agentes puedan realizar las siguientes funciones.


\subsection{Comienzo del turno de servicio.}
Cuando el agente desee comenzar un nuevo turno de servicio deberá introducir los datos relacionados con el mismo. Estos datos son los siguientes:

\begin{itemize}
	\item 
\end{itemize}


\subsection{Crear incidencias.}
\subsection{Terminar el turno de servicio.}


\section{Desarrollo del sistema de croquis.}


\section{Tests unitarios.}

\section{Integración continua.}

\section{Instalación de la aplicación de manera automática.}


