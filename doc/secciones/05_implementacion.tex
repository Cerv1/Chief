\chapter{Implementación}

La implementación del software se ha dividido en hitos. Estos, han sido definidos en Github
y cada uno de ellos contiene un grupo de \textit{issues} que se corresponden con las distintas
mejoras que ha ido incorporando el software a lo largo de su desarrollo.\\

Los hitos han sido los siguientes en orden cronológico:

\begin{enumerate}
	\item \textbf{Permitir registros y acceso de usuarios.}
	\item \textbf{Desarrollo del sistema de documentación.}
	\item \textbf{Desarrollo del sistema de croquis.}
	\item \textbf{Desarrollo del sistema de tests unitarios.}
	\item \textbf{Integración continua.}
	\item \textbf{Instalación de la aplicación de manera automática.}
\end{enumerate}

Toda la información relacionada con la implementación y el desarrollo se pueden encontrar en:

\begin{itemize}
	\item \href{https://github.com/Cerv1/Chief/milestones}{Lista de hitos.}
	\item \href{https://github.com/Cerv1/Chief/commits}{Lista de commits.}
\end{itemize}

\newpage

\section{Registro y acceso de usuarios.}
El objetivo de este hito es permitir que un usuario agente cree una cuenta en la aplicación 
y posteriormente pueda acceder a la misma si la contraseña y el usuario es válido. \\

Para poder utilizar el servidor de una manera más cómoda y con un mayor número de funcionalidades se ha utilizado
\textbf{Express.js}\cite{express} para crear el servicio web que se ofrece. De esta manera, conseguimos utilizar el 
servidor de una manera más eficiente.\\

Express.js ofrece un software liviano que dota de funcionalidades como negociación de contenido, una mejora en el 
rendimiento del servidor, la posibilidad de seleccionar un motor de plantillas para servir el contenido... En este caso,
se ha configurado para que utilice como motor de plantillas el ya mencionado \textbf{Pug}. De esta manera, cuando el usuario
realice una petición sobre alguna ruta se renderizará dicha plantilla para que el usuario reciba el archivo HTML, el cual
será interpretado por el navegador que esté utilizando.\\ 

Un usuario que quiera acceder a la aplicación o a cualquiera de las funcionalidades que ofrece deberá estar autenticado.
Para administrar de una manera más segura los datos de sesión de un usuario se ha utilizado la librería \textbf{Passport.js}\cite{passport}.
Esta librería se encarga de autenticar las peticiones hacia el servidor a través de una serie de \textit{plugins} conocidos
como \textit{strategies}. Gracias al comportamiento de esta librería podemos gestionar de una manera sencilla el acceso a los usuarios.
En este caso, se ha establecido un inicio de sesión mediante \textbf{usuario} y \textbf{contraseña} definidos por el agente en el 
momento de registrarse.\\

Cuando un agente realiza un registro, sus datos se almacenan \textbf{encriptados} en la base de datos para que de ninguna manera
se puedan alterar o recuperar de forma fraudulenta. El cifrado de dichos datos se realizan con un algoritmo basado en el cifrado
Blowfish\cite{blowfish} al que se le añaden unos bits aleatorios para proteger nuestro cifrado ante un ataque de Tabla Arcoíris\cite{rainbow}.


\section{Desarrollo del sistema de documentación.}

El primer objetivo de este hito es conseguir un sistema de incidencias completamente funcional y la posibiidad de generar todos los documentos
soportados por la plataforma de manera interactiva y desde cualquier localización. De esta manera se consigue que los agentes aumenten su 
productividad ya que pueden rellenar los documentos de manera inmediata y no tendrán que realizarlo al final del turno.\\

Para la realización de este hito se ha utilizado un módulo completamente propio llamado \textbf{doc\textunderscore functions}. Este módulo ha sido diseñado para ser usado
como biblioteca externa ya que contiene un gran número de funciones para generar los distintos tipos de documentos que podrá manejar la aplicación. De esta manera
se consigue una mejor reutilización del código, además de lograr un menor acoplamiento.\\

Esta biblioteca se encarga de crear el dato de tipo JSON que será posteriormente inyectado en las plantillas. Para ello, tiene en cuenta los datos recibidos
del servidor y otros que gestiona de manera independiente al usuario como pueden ser la fecha, la hora en la que se crea el turno, el formato de salida del 
documento... Además tiene otra función importante y es la de dar feedback al usuario. En la función en la que crea el documento se he creado un \textit{callback}
para poder comunicarse de nuevo con el servidor. De este modo, se podrá saber si la creación del documento ha sido correcta o, por el contrario, si ha fallado.
Podremos utilizar dicho \textit{callback} para devolver un \textit{feedback} al usuario y que sea consciente del resultado de la operación. Por tanto, todas las 
acciones de escritura y creación de documentos han sido desplazadas a dicho módulo, logrando así un mejor diseño y un código mejor estructurado.
\\

Una vez presentado el módulo creado para este hito, se procederá a detallar las funciones que se han creado para el usuario.

\subsection{Comienzo del turno de servicio.}
Cuando el agente desee comenzar un nuevo turno de servicio deberá introducir los datos relacionados con el mismo. Estos datos son los siguientes:

\begin{itemize}
	\item \textbf{Turno.} Podrá ser mañana, tarde o noche.
	\item \textbf{Identificadores de los policías.} Puede ser desde 1 hasta 4 los policías que estén en un turno simultáneamente.
	\item \textbf{Jefe de guardia.} El nombre del agente que actúa como supervisor del turno.
\end{itemize}

Una vez se rellenan estos campos, se creará un nuevo fichero de turno y se rellenarán las cabeceras con los datos anteriormente solicitados.

\subsection{Crear incidencias.}

En este caso, el agente podrá crear un nuevo registro perteneciente a una incidencia que será guardado en la base de datos. Para crear una nueva
incidencia se deberán rellenar los siguientes campos:

\begin{itemize}
	\item \textbf{Número de incidencia.} Número para identificarla. El agente decide que código le asigna.
	\item \textbf{Dependencia.} Dependencia desde la que se ha realizado la incidencia. Por ejemplo , alguien que llama por teléfono o una llamada de la central.
	\item \textbf{CEN.} Identificador del PREGUNTAR.
	\item \textbf{PP.LL.} Número del agente que registra la incidencia.
	\item \textbf{Hecho.} Descripción del motivo de la incidencia.
	\item \textbf{Actuación.} Acutación desempeñada por los agentes.
\end{itemize}

\subsection{Terminar el turno de servicio.}

Una vez terminado el servicio, se puede proceder a cerrar el parte de servicio. Cuando el agente realiza esta acción, \textbf{Chief} accede 
a la base de datos interna y genera un documento final. Este documento tendrá un nombre con el siguiente formato: \textit{DD\textunderscore MM\textunderscore YYYY\textunderscore Turno}.\\	

De esta manera, identificaremos a los partes de incidencias con un día, mes, año y nombre del turno. Es así debido a que no podrá haber más de un fichero de fin 
de turno para el mismo turno. Con este sistema se crean dichos documentos pero sin poder modificar las incidencias. Los registros que aparecen en el documento
final son todos los que han ocurrido dentro del intervalo del turno.

\subsection{Desarrollo del sistema de ordenanzas y accidentes.}

Una vez realizado el hito anterior, ya tenemos una aplicación con su sistema de identificación para los agentes y además pueden realizar todas las acciones
relacionadas con los partes de servicio habituales. Por tanto, faltaría incluir el sistema de accidentes y ordenanzas para completar el desarrollo de la
creación de documentos administrativos.\\

Todos los documentos serán generados en una carpeta específica donde serán agrupados todos los tipos de informes generados. En este caso, el nomrbe de los ficheros se
genera con el siguiente formato: \textit{DD\textunderscore MM\textunderscore YY\textunderscore HH:MM}.\\

Los documentos que el agente puede registrar son los siguientes:

\begin{itemize}
	\item \textbf{Ordenanza de limpieza. } Se generará un acta de denuncia ante la ordenanza de limpieza.
	\item \textbf{Ordenanza de ruidos en la vía pública. } Se generará un acta de denuncia por ruidos u otras actividades molestas en en la vía pública bajo el artículo número
	43.2, del Decreto 326/03 y la Ley 7/06 de 24 de Octubre. REFS NEEDED.
	\item \textbf{Ordenanza de ruidos en domicilio. } Se generará un acta de denuncia con el artículo elegido por el agente de la Ley 37/2003 y bajo el Decreto 326/03. REFS NEEDED.
	\item \textbf{Ordenanza de ruidos en local. }  Se generará un acta de denuncia con el artículo elegido por el agente de la Ley 37/2003 y bajo el Decreto 326/03. REFS NEEDED.
	\item \textbf{Acta de medición de ruidos. } Se generará un acta de medición de ruidos bajo el artículo elegido por el agente de la Ley 37/2003 y bajo el Decreto 326/03. REFS NEEDED.
	\item \textbf{Acta de inspección de obras. }  Se generará un acta de inspección de obras y quedará numerada bajo el número de acta elegido por el agente.
	\item \textbf{Acta de hallazgo de residuos. } Se generará un acta de denuncia en la que queda constancia de que se dará parte a la Concejalía de Medioambiente y Urbanismo
	\item \textbf{Accidente de 2 vehículos. } Se creará un diligencia en la que quedarán registrados los datos de todos los involucrados en el accidente, tanto testigos como 
	conductores de los vehículos.
	\item \textbf{Accidente de 3 vehículos. } Se creará un diligencia en la que quedarán registrados los datos de todos los involucrados en el accidente, tanto testigos como 
	conductores de los vehículos.
\end{itemize}

\section{Desarrollo del sistema de croquis.}

\section{Tests unitarios.}

\section{Integración continua.}

\section{Instalación de la aplicación de manera automática.}


