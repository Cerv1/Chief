\chapter{Conclusiones y trabajos futuros}

Este desarrollo se ha completado utilizando exclusivamente recursos libres, tanto la información consultada a lo largo del proyecto, como 
las herramientas y bibliotecas necesarias. De esta manera, se ha conseguido crear un software \textbf{completamente libre} que puede
ser usado para instituciones del gobierno, como en este caso, un Ayuntamiento. \\

De igual manera que se ha completado satisfactoriamente el objetivo de ayudar a los agentes de policía. Aunque este desarrollo abarca hasta 
una fase temprana de la aplicación, los agentes han quedado muy satisfechos con el contenido que proporciona esta versión y en el entorno
de pruebas ha sido todo un éxito.\\

Por otro lado, se ha aprendido a gestionar un proyecto a gran escala desde 0. El desarrollo podría haber sido más rápido si no se hubieran 
incluido los \textit{test unitarios} o el \textit{despliegue y provisionamiento} pero la calidad del código hubiera disminuido. Incluyendo
esta metodología, podemos garantizar que nuestra aplicación ha sido probada ante un gran número de entradas, demostrando la robustez del sistema. 
De igual modo, la realización de un sistema de aprovisionamiento y despliegue garantiza la portabilidad y usabilidad. Esto se debe a que
un usuario no tiene que preocuparse de configurar el entorno, sino que directamente se crea y se configura de manera automática. Así, conseguimos
acortar el tiempo que se tarda en desplegar \textbf{Chief} tanto en local para realizar pruebas de desarrollo como en la nube para pasar el estado de
producción.\\

El diseño de esta aplicación es un proyecto suficientemente complejo y enfocado al mundo real donde. Ha sido creada íntegramente para garantizar la satisfacción del cliente. Esto se traduce de manera directa en las funcionalidades presentes.\\

Además, como procedimiento para asegurar la calidad de dicho trabajo, se han realizado pruebas a nivel de software para evitar los errores derivados de fallos en la implementación o en el diseño de la aplicación. Así logramos crear un producto a prueba de errores y robusto antes de avanzar a la fase de producción como vimos en la \autoref{tests-section}.\\


Para futuros trabajos, sin duda alguna, se desplegaría la aplicación en un sistema en producción donde los agentes puedan utilizarlo en su día
a día. Además, se seguirían añadiendo nuevas funcionalidades como un sistema de localización en tiempo real, la posibilidad de gestionar los cuadrantes de los agentes e incorporar nuevos documentos para ayudar a los agentes a servir de una manera más efectiva a los ciudadanos.