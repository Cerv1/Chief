\chapter{Planificación}
\section{Metodología utilizada}

Para el desarrollo de este trabajo se ha utilizado una metodología ágil y, en concreto, se ha optado por una versión personal basada en \textit{Scrum}\cite{scrum}. En esta metodología
ágil propia, se sigue un desarrollo por pequeñas funcionalidades que se deben resolver en pequeños periodos de tiempo llamados \textit{sprints}. Para organizar los contenidos, así como las tareas
a realizar y el progreso de cada una, se ha utilizado la funcionalidad de \textbf{Proyectos} de GitHub, en la cual las tareas se han dividido en 3 secciones. Estas tareas
quedan representadas por \textit{issues}, que se utilizan para llevar un progreso del desarrollo. La lista completa de \textit{issues} se puede encontrar \href{https://github.com/Cerv1/Chief/issues?q=is\%3Aissue+is\%3Aclosed}{aquí.} \\

Estos \textit{issues} estarán dentro del proyecto en uno de los 3 estados siguientes:

\begin{itemize}
	\item \textbf{Por hacer.} Aún no se ha comenzado el desarrollo de la nueva funcionalidad, el arreglo de un fallo o la actividad a la que referencie el \textit{issue}.
	\item \textbf{En progreso.} Actualmente se está trabajando para terminar el objetivo por el cual ha sido abierto el \textit{issue}.
	\item \textbf{Finalizado. } El desarrollo correspondiente a la actividad ha sido terminado y se ha incorporado al código fuente de la aplicación.
\end{itemize}

Además, estos \textit{issues} se han agrupado dentro de \textit{milestones} o hitos, los cuales serán definidos en capítulos posteriores.\\

Este tipo de metodología ayuda a conseguir un progreso coherente y más acertado. Esto se debe a que es más fácil estimar la cantidad de tiempo que se requerirá para
implementar una pequeña funcionalidad o solucionar un error que incluir un sistema completo y muy complejo. Otra ventaja de seguir esta metodología es que se puede ver
un desarrollo realista de la aplicación desde el principio hasta el final. Se puede comprobar en qué punto se incluyó exactamente qué funcionalidad y cuanto tiempo de 
trabajo requirió. 

\section{Temporización}
La planificación temporal que se ha establecido durante el desarrollo ha sido la siguiente:

\renewcommand{\arraystretch}{1.5}\label{time-section}
\begin{table}[H]
	\centering
	\label{tabla-temporal}
	\resizebox{\textwidth}{!}{%
	\begin{tabular}{@{}ccc@{}}
		\toprule
		\rowcolor[HTML]{ECF4FF} 
		\textbf{Etapas del desarrollo}                                                                                                                  & \textbf{Fecha de comienzo} & \textbf{Fecha de finalización} \\ \midrule
		\cellcolor[HTML]{ECF4FF}\textbf{\begin{tabular}[c]{@{}c@{}}Reuniones iniciales para realizar\\ las especificaciones del proyecto.\end{tabular}} & 19 de Marzo                & 2 de Abril                     \\
		\rowcolor[HTML]{EFEFEF} 
		\cellcolor[HTML]{ECF4FF}\textbf{Planificación de los contenidos}                                                                                & 3 de Abril                 & 10 de Abril                    \\
		\cellcolor[HTML]{ECF4FF}\textbf{\begin{tabular}[c]{@{}c@{}}Análisis y diseño tanto del problema\\ como de las soluciones.\end{tabular}}         & 11 de Abril                & 22 de Abril                    \\
		\rowcolor[HTML]{EFEFEF} 
		\cellcolor[HTML]{ECF4FF}\textbf{Implementación del software.}                                                                                   & 22 de Abril                & 1 de Junio                     \\
		\cellcolor[HTML]{ECF4FF}\textbf{Pruebas en un entorno controlado.}                                                                              & 1 de Junio                 & 7 de Junio                     \\
		\rowcolor[HTML]{EFEFEF} 
		\cellcolor[HTML]{ECF4FF}\textbf{Documentación.}                                                                                                 & 2 de Junio                 & 15 de Junio                    \\ \bottomrule
	\end{tabular}}
	\caption{Tabla con la organización temporal del proyecto.}
\end{table}

Como se puede comprobar, el proceso de elaborar esta documentación comenzó al mismo tiempo que las pruebas en un entorno controlado. Esto se debe a que a partir de este punto, 
no se iba a incorporar más funcionalidad al software y por tanto, se podía comenzar a planear la documentación, preparar las plantillas de LaTeX y organizar el contenido.

\section{Seguimiento del desarrollo}
Para llevar un correcto seguimiento del desarrollo, se ha planificado una reunión todos los Lunes con el tutor de dicho trabajo, D. Juan Julián Merelo. En estas pequeñas 
tutorías se presentaban las mejores y se comentaba el trabajo a desarrollar durante la siguiente semana. Además, su opinión ha sido de gran utilidad a la hora de descubrir
nuevas herramientas o utilizar las ya existentes de una mejor manera.\\

Gracias a estas reuniones se ha podido llevar un seguimiento del proceso de creación, desde el comienzo hasta el final de su etapa.
