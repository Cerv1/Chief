\chapter{Descripción del problema}
Los objetivos del desarrollo de este proyecto son los siguientes:
\begin{enumerate}
    \item \textbf{Facilitar el trabajo a los policías locales de Maracena.}\\
    Debido a la inclusión de un sistema informatizado para la elaboración de 
    ficheros administrativos, gestión de incidentes y un sistema de croquis. Consiguiendo,
    por tanto, una mejora en la productividad de los agentes y un aumento de la seguridad
    global de los datos almacenados.
    
    \item \textbf{Inclusión del software libre en organismos del estado.}\\
    Dejando a un lado herramientas privativas sobre las que no tenemos un pleno control
    de los datos que están analizando y además fomentando el desarrollo libre. Porque 
    de esta manera cualquier persona puede sumarse al desarrollo y mejora de este software.

    \item \textbf{Elaboración de un entorno de pruebas real.}\\
    Se persigue la implementación de un entorno virtual lo más parecido posible a la 
    realidad para que el uso de la aplicación pueda ser probado antes del despliegue 
    final. 
   
    \item \textbf{Probar los conocimientos adquiridos a lo largo del grado.}\\
    Creando un sistema completo en el que tendremos que tener en cuenta todos los 
    aspectos técnicos adquiridos en el transcurso de la carrera. Dada la embergadura del
    proyecto, es necesario disponer de una base muy consolidada de los conocimientos adquiridos
    previamente.

    \item \textbf{Aprendizaje de tecnologías punteras en el sector.}\\
    Durante el desarollo del proyecto se ha buscado aprender y utilizar tecnologías en 
    auge que tienen un alto potencial. Esta decisión se toma en base  a la cantidad de gente
    y empresas que las utilizan, la comunidad tan amplia que tienen así como las contínuas
    mejoras que están recibiendo.  

\end{enumerate}
